% Author: Leo Büttiker bassed on code from Till Tantau
% Source: Based on The PGF/TikZ manual
%%\documentclass{article}

\documentclass[a4paper,12pt]{article}

%%Include tikz, for drawing
\usepackage{tikz}
\usetikzlibrary{shapes,arrows}


\input{./includes/standard-definitions.tex}
\input{./includes/standard-exam-summary.tex}
\input{./includes/asmthm_de.tex}

%Overwrite std def
\geometry{top=3cm,left=5cm,right=5cm}

\begin{document}
\pagestyle{empty}

\section{Differentialrechnung in $\R^n$}

\subsection{Kritische Punkte}
Berechnen von lokalen Minima und Maxima. (Achtung: Rand separat betrachten!)\\
\\
%\resizebox{!}{15cm}{%
%tikz graph inspiered by http://slopjong.de/2010/04/10/urinal-manual/
%content inspired by jörn http://www.youtube.com/watch?v=fdDuptml-2E
\tikzstyle{decision} = [diamond, draw, fill=blue!20, 
text width=4.5em, text badly centered, node distance=3cm, inner sep=0pt]

\tikzstyle{block} = [rectangle, draw, fill=blue!20, 
text width=4cm, text centered, rounded corners, minimum height=4em, minimum width=4.5cm]

\tikzstyle{line} = [draw, -latex']

%from http://tex.stackexchange.com/questions/60585/how-to-highlight-a-single-element-in-a-matrix
\newcommand\hlight[1]{\tikz[overlay, remember picture,baseline=-\the\dimexpr\fontdimen22\textfont2\relax]\node[rectangle,fill=yellow!50,rounded corners,fill opacity = 0.2,draw,thick,text opacity =1] {$#1$};} 


\begin{tikzpicture}[node distance = 3cm, auto]
% The nodes  
\node[block] (grad) {Grad(f) = 0 ergibt kritische Punkte ($\nabla f = \vec{0}$)};    
\node[block, below of=grad] (matrix) {Hesse-Matrix $\operatorname{H}_f$ berechnen};    
\node[block, below of=matrix] (det) {Determinante Berechnen $a_{11} a_{22} - a_{12} a_{21}$};    
\node [decision, below of=det, text width=2cm] (detS) {$det < 0$?};
\node [decision, below of=detS, text width=2cm] (detB) {$det > 0$?};
\node [decision, below of=detB, text width=2cm] (definit) {$a_{11} > 0$?};

\node[block, right of=detS, node distance=5cm] (sattel) {Sattelpunkt};
\node[block, right of=detB, node distance=5cm] (unb) {Unbestimmt $det=0$};
\node[block, right of=definit, node distance=5cm] (max) {Lokales Maximum};
\node[block, below of=definit] (min) {Lokales Minimum};

\node[rectangle, right of=matrix , node distance=5cm] (hf) {$\operatorname{H}_f=
\begin{pmatrix}
\hlight{\frac{\partial^2 f}{\partial^2 x}}&\frac{\partial^2 f}{\partial x\partial y}\\
\\
\frac{\partial^2 f}{\partial y\partial x}&\frac{\partial^2 f}{\partial^2 y}
\end{pmatrix}$};


% The connections
\path [line] (grad) -- node [near start] {$(x_0,y_0)$} (matrix);

\path [line] (matrix) -- node [near start] {$\operatorname{H}_f$} (det);
\path [line] (det) -- node [near start] {$det$} (detS);
\path [line] (detS) -- node [near start] {Ja} (sattel);
\path [line] (detS) -- node [near start] {Nein} (detB);

\path [line] (detB) -- node [near start] {Nein} (unb);
\path [line] (detB) -- node [near start] {Ja} (definit);

\path [line] (definit) -- node [near start] {Nein} (max);
\path [line] (definit) -- node [near start] {Ja} (min);
\end{tikzpicture}
%}


\end{document}