% Author: Leo Büttiker bassed on code from Till Tantau
% Source: Based on The PGF/TikZ manual
%%\documentclass{article}

\documentclass[a4paper,12pt]{article}

\usepackage{tikz}
\usetikzlibrary{mindmap,trees}


\input{./includes/standard-definitions.tex}
\input{./includes/standard-exam-summary.tex}
\input{./includes/asmthm_de.tex}


\begin{document}
\pagestyle{empty}
\begin{tikzpicture}
  \path[mindmap,concept color=black,text=white]
    node[concept] {Analysis}
    [clockwise from=0]
    child[concept color=green!50!black,level distance=6cm] {
      node[concept] {Extremwerte}
      [clockwise from=100]
      child { node[concept] {Funktionen in $\R^n$} 
	 child { node[concept] (hesse){Hesse Matrix } }
	 child { node[concept] {Gradient $\nabla f = 0$} }
      }
      child { node[concept] {Extremstellen berechnen $f'=0$} }
      child { node[concept] {Minimum und Maximum}  
	 [clockwise from=45]
     	child { node[concept] {$f''(x_0)>0$ lokales minima} }
	child { node[concept] {$f''(x_0)<0$ lokales maxima} }
	child { node[concept] {$f''(x_0)=0$ Sattelpunkt} } }
      child { node[concept] {Wendepunkt $f''=0$} }
      child { node[concept] {Nullpunkt} }
      child { node[concept] {Grenzwerte von f nach  $\pm\infty$} }
    }  
    child[concept color=blue] {
      node[concept] {Beweise}
      [clockwise from=-30]
      child { node[concept] {Vollständige Induktion} }
      child { node[concept] {...} }
    }
    child[concept color=red] { node[concept] {Diff'gleichung} }
    child[concept color=orange] { node[concept] {Integrale}
      [clockwise from=220]
      child { node[concept] {Riemann Summe} }
      child { node[concept] {Integrale mit Partialbruchzerlegung} }
    }
    child[concept color=purple] { node[concept] {Konvergenz}
      [clockwise from=170]
      child { node[concept] {Konvergenz-radius} }
      child { node[concept] {Limes} 
	[clockwise from=185]
	 child { node[concept] {$\lim_{n \to \infty} a_n = a$} }
	 child { node[concept] {$\epsilon$-$\delta$-Kriterium} }
      }
      child { node[concept] {Funktionen} 
	[clockwise from=100]
	 child { node[concept](pktKonv) {Punktweise Konvergenz} }
	 child { node[concept](glmKonv) {Gleichmässige Konvergenz} }	
      }
      child { node[concept] {Reihen} }
    }
    child[concept color=yellow] { node[concept](taylor) {Taylorreihen} };

    \node[annotation,left,left color=purple!50!black] at (pktKonv.west)
	{
		$\forall x \in D_f: \lim_{n \to \infty} f_n(x) = a$
	};

    \node[annotation,right,left color=purple!50!black] at (glmKonv.east)
	{
		$\forall x \in D_f \forall \epsilon>0 \exists N \in \N \forall n \geq \N: |f_n(x) - f(x)| < \epsilon$
	};
    \node[annotation,left,left color=yellow] at (taylor.north)
	{
		$\sum_{n=0}^\infty \frac{f^{(n)}(a)}{n!} (x-a)^n$
	};

    \node[annotation,left,left color=green!50!black] at (hesse.north)
      {$H_f = 
	\begin{pmatrix}
		\frac{\partial^2 f}{\partial^2 x} & \frac{\partial^2 f}{\partial x\partial y} \\
		\frac{\partial^2 f}{\partial y\partial x} & \frac{\partial^2 f}{\partial^2 y} 
	\end{pmatrix}$};
\end{tikzpicture}\end{document}